\section{Background of the Study}

 Search and rescue mobile robots have been widely used in different situations mostly pertaining to disasters such as earthquakes, fire, and, urban accidents, mining accidents, and etc. The main purpose of these robots is to locate the victims or survivors in unreachable or dangerous areas. Once the survivors are located, rescue teams can now start the process of rescuing the survivor with more efficiency due to the know location of the said survivor. These mobile bots are either controlled through a remove device or self-operated through a programmed algorithm.  Some search and rescue mobile bots have the capability of rescuing the survivor themselves but due to their large size, they are at a disadvantage in rescue operations where in survivors are under a pile of rubble and are fragile and need careful handling. Small mobile bots are more used than larger ones. The advantages of using these small search and rescue mobile bots are reduced fatigue, less man power, efficiency, and access to unreachable locations. 

 Mobile robots include components such as a DC motor or a servomotor, motor driver to supply the required power the motor needs, a microcontroller such as the Arduino to serve as its core unit, and a battery. Components such as lights, sensors, and signal modules are added to make the mobile bot more convenient to certain operations. The most important part of a mobile bot is the microcontroller because it serves as the brain of the motor. The microcontroller is programmable using C language. The behavior of the mobile bot depends on the algorithm programmed in the microcontroller. Mobile bots can be controlled by a remote control device such as a smartphone or a laptop. There are different kinds of signals used to transmit signals between the remote control and the mobile bot. These signals can be Bluetooth, RF signal, Wi-Fi signal, ZigBee, and etc. One of the methods in this option of project is using bluetooth. Bluetooth transmits data through low-power radio waves, communicating at a frequency of around 2.45 gigahertz. The limitation of bluetooth is its low power method of transmitting signals typically 1 milliwatts, this greatly limits the range of bluetooth to around 10m. Another option for wireless communication aside from bluetooth, is the radio frequency (RF) which are electromagnetic wave frequencies that lie in the range of 3 khz to 300 Ghz. The problem our group has in using rf is its limitation in bandwidth, in our project we also need to stream a live video feedback from the robot. As for wireless communication we settled for wifi. The reason for choosing wifi is because of its ability to satisfy our need for continous video feedback with controls and all this could be done with wifi. 


\section{Prior Studies}


	On other studies that revolve and focus as well on remote controlled mobile robots, internet connection is being used so that it can be controlled for any given location. On this study, the remote controlled mobile robot is going to be developed alongside the use of Bluetooth connection. This is to enable controlling the mobile robot even without the use of internet connection. A rotating camera as well is being implemented on this study to enable surveying and scouting of areas that the robot would be going into. 

\section{Problem Statement}

	The problem that our thesis will address is that it usually takes a lot of time for and effort to conduct search and rescue operations if it would be done physically and it will consume a lot of energy from the rescue team. It is because of this problem that we are suggesting the use of an android controlled wifi mobile robot with an attached camera to conduct this search and rescue operations because it would ease the work for the rescue team that will conduct the operations. They would already have a head start on the area where they will concentrate their search which means that the chances of finding and rescuing the people are greater because the area to conduct the search has been minimized. Also, through the camera of the mobile robot, the rescue team are able to have a live view of what is actually taking place on the site where the robot is conducting the operations.


\section{Problem Statement}
\blindtext




\section{Objectives}
\subsection{General Objective(s)}
To \ldots;

\subsection{Specific Objectives}

\begin{enumerate}
	\item To  \ldots;
	
	\item To  \ldots;
	
	\item To  \ldots;
	
	\item To  \ldots;
	
	\item To  \ldots;
\end{enumerate}



\section{Significance of the Study}


	The significance of this study is that it caters to the safety of the public because the rescue process is expedited and the time it takes for them to be rescued will be lesser compared to conducting the operations without the use of the mobile robot. In addition, the study will also enhance the rescue operation process since it has the technology in order to map out the geographical conditions of the area through the use of its attached camera. It will be able to see if there are survivors that are perhaps trapped or still in distress and the rescue team will be able to take the immediate action thereafter.



\section{Assumptions, Scope and Delimitations}

Bulletize your scope in one group, and then bulletize the delimitations in another.  Bulletize your assumptions as well.


\section{Description and Methodology}



\ifFinished
\else

\section{Estimated Work Schedule and Budget}

Gantt chart or similar is to be part of this section.


\section{Publication Plan}

\fi


\section{Overview}




